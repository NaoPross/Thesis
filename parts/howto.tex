% vim: set ts=2 sw=2 et tw=78:

\chapter{How to use this template}


\section{Requirements and typesetting}

This document is made to be compiled using \XeLaTeX{} or \LuaLaTeX{} and
\LaTeXe. Do not compile it using \texttt{pdflatex} or very old \LaTeX
versions, it will not work! The bibliography is generated with
\texttt{biblatex} using the \texttt{biber} backend. If you work with the
command line you need to execute the following commands:
\begin{verbatim}
  $ xelatex Main.tex
  $ biber Main.tex
  $ xelatex Main.tex
\end{verbatim}
Though this will generate a lot of auxiliary files everywhere so I recommend
using \texttt{cluttex} instead:
\begin{verbatim}
  $ cluttex -e xelatex --biber Main.tex
\end{verbatim}
Which will run \XeLaTeX{} (or \LuaLaTeX{}) the right number of times and
produce just a single PDF, while storing the auxiliary files somewhere else in
a temporary directory.

\section{Directory structure}

The structure of this template in the filesystem is as follows:
\begin{verbatim}
  $ tree --charset ascii Thesis
  Thesis
  |-- Main.bib
  |-- Main.pdf
  |-- Main.tex
  |-- appendix
  |-- figures
  |   `-- galactic-empire.png
  |-- parts
  |   |-- abstract.tex
  |   `-- howto.tex
  `-- tex
      |-- docmacros.sty
      |-- hsrthesis.cls
      `-- titlepage.tex
\end{verbatim}
There are two files called \texttt{Main} that may be renamed to something more
pertinent to your thesis, though note that the new name cannot contain spaces.
The \texttt{tex} directory only contains the ``code'', that means styling,
macros and other special sources such as the title page.  The content of your
thesis should be stored under the \texttt{parts} directory, \texttt{Main.tex}
should not contain any text except for the title, date and the authors.  You
are free to split your text into however many files you like, though I
recommend one file per chapter.  The \texttt{figures} folder and
\texttt{appendix} are rather self explanatory, they contain the figures and
the content of the appendix respectively.

\section{The document macros file}

In the style file \texttt{tex/docmacros.sty} there are a few help macros might
be useful to electrical engineers. Feel free to look and tweak it to your
liking. For now I have included:
\begin{itemize}
  \item Probability operators
    \lstinline!\E{X} \Pr{X} \Var{X}! (\(\E{X} \Pr{X} \Var{X}\)).

  \item Special functions
    \lstinline!\sinc \argmax \argmin! (\(\sinc \argmax \argmin\)).

  \item Redefined the imaginary and real operator \lstinline!\Re{z} \Im{z}!
    (\(\Re{z} \Im{z}\)) to look ``normal'', i.e. not with Fraktur font.

  \item Linear operators \lstinline!\laplace \fourier \hilbert!
    (\(\laplace \fourier \hilbert\)).

  \item Redefined vector notation \lstinline!\vec{u}! (\(\vec{u}\)) to use a
    bold font, and a new unit vector macro \lstinline!\uvec{u}!
    (\(\uvec{u}\)).

  \item Bold notation for the dot product \lstinline!\dotp! (\(\dotp\)) and
    cross product \lstinline!\crossp! (\(\crossp\)), to distinguish them from
    \lstinline!\cdot! (\(\cdot\)) and \lstinline!\times! (\(\times\)).

  \item Bold upright matrix notation \lstinline!\mx{A}! (\(\mx{A}\)).
\end{itemize}
